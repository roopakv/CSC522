\documentclass[fontsize=10pt]{scrartcl}

\usepackage{enumitem}
	\setenumerate{listparindent=\parindent}
\usepackage{amsmath}
\usepackage{amssymb}
\usepackage{graphicx}
\usepackage{placeins}
\usepackage{hyperref}

\usepackage[margin=1.0in]{geometry}

\newcommand{\code}{\texttt}

\begin{document}

	\title{CSC 522 : Automated Learning and Data Analysis}
	\subtitle{Homework 2}
	\author{Roopak Venkatakrishnan - rvenkat7@ncsu.edu}
	\maketitle

	\section{Question 1}
		Write code in R or Matlab to perform each of the following tasks:
		\begin{enumerate}
			\item
			Generate a $3 \times 3$ matrix with input containing the sequence 1, 2, ... 9. \\
			\textbf{Ans:} $x \leftarrow matrix(c(1:9),3,3)$
			\item
			\begin{enumerate}
				\item
				Access elements from the 2nd and 3rd columns only. \\
				\textbf{Ans:} \\
				2nd column alone : x[,2] $\implies$ [1] 4 5 6\\
				3rd column alone : x[,3] $\implies$ [1] 7 8 9\\ 
				both columns     : x[,2:3] \\
				$\implies$ \\
				
				$\begin{array}{ccc} 
					~		&	[,1]	&	[,2]	\\ \relax
					[1,]	&	4	 	&	7		\\ \relax
					[2,]	&	5		&	8		\\ \relax
					[3,]	&	6		&	9		\\ \relax
				\end{array}$

				
				\item
				Access elements of the 2nd and 3rd rows only \\
				\textbf{Ans:} \\
				2nd row alone : x[2,] $\implies$ [1] 2 5 8\\
				3rd row alone : x[3,] $\implies$ [1] 3 6 9\\ 
				both rows     : x[2:3,] \\
				$\implies$ \\
				$\begin{array}{cccc}
					~		&	[,1]	&	[,2]	&	[,3]	\\ \relax
					[1,]	&	2		&	5		&	8		\\ \relax
					[2,]	&	3		&	6		&	9		\\ \relax
				\end{array}$


				\item
				Access rows 1 and 3 only? (see rbind() function in R and vertcat() in matlab) \\
				\textbf{Ans:} \\
				$x2 \leftarrow rbind(x[2,],x[3,])$ \\
				$\implies$ \\
				$\begin{array}{cccc}
					~		&	[,1]	&	[,2]	&	[,3]	\\ \relax
					[1,]	&	1		&	4		&	7		\\ \relax
					[2,]	&	3		&	6		&	9		\\ \relax
				\end{array}$

				\item
				Calculate sum of the 2nd row, the diagonal and the 3rd column in the matrix. \\
				\textbf{Ans:} \\
				x[2,] + x[,3] + diag(x) \\
				$\implies$  [1] 10 18 26 \\

				\item
				Identify row and column dimensions of the matrix.\\
				\textbf{Ans:} \\
				dim(x) \\
				$\implies$ [1] 3 3 \\

				\item
				Transpose of a matrix. \\
				\textbf{Ans:} \\
				t(x) \\
				$\begin{array}{cccc}
				~		&	[,1]	&	[,2]	&	[,3]	\\ \relax
				[1,]	&	1		&	2		&	3		\\ \relax
				[2,]	&	4		&	5		&	6		\\ \relax
				[3,]	&	7		&	8		&	9		\\ \relax
				\end{array}$

				\item
				Scalar multiplication of output matrix with itself. \\
				\textbf{Ans:} \\
				x * x \\
				$\begin{array}{cccc}
				~		&	[,1]	&	[,2]	&	[,3]	\\ \relax
				[1,]	&	1		&	16		&	49		\\ \relax
				[2,]	&	4		&	25		&	64		\\ \relax
				[3,]	&	9		&	36		&	81		\\ \relax
				\end{array}$

				\item
				Matrix multiplication of output matrix with itself. \\
				\textbf{Ans:} \\
				x \%*\% x \\
				$\begin{array}{cccc}
				~		&	[,1]	&	[,2]	&	[,3]	\\ \relax
				[1,]	&	30		&	66		&	102		\\ \relax
				[2,]	&	36		&	81		&	126		\\ \relax
				[3,]	&	42		&	96		&	150		\\ \relax
				\end{array}$

				\item
				Cross product of the output matrix from 1. \\
				\textbf{Ans:} \\
				crossprod(x) \\
				$\begin{array}{cccc}
				~		&	[,1]	&	[,2]	&	[,3]	\\ \relax
				[1,]	&	14		&	32		&	50		\\ \relax
				[2,]	&	32		&	77		&	122		\\ \relax
				[3,]	&	50		&	122		&	194		\\ \relax
				\end{array}$

				\item
				Check if a matrix is a square matrix. \\
				\textbf{Ans:} \\
				\begin{verbatim}
function checksqmatrix(mat) 
{ 
  if(dim(mat)[1]==dim(mat)[2]) 
  { 
    print("It is a square matrix!") 
  } 
  else 
  { 
    print("It is NOT a square matrix") 
  } 
} 

> checksqmatrix(x)
[1] "It is a square matrix!"

> checksqmatrix(matrix(c(1:10),2,5))
[1] "It is NOT a square matrix"
\end{verbatim}
				
				\item
				Inverse of a matrix \\
				\textbf{Ans:} \\
				solve(x) \\
				Since this matrix has determinant 0 the inverse is not defined. \\
				Error in solve.default(x) :  \\
  				Lapack routine dgesv: system is exactly singular: U[3,3] = 0 \\

  				\item
  				Identity of a matrix. \\
  				\textbf{Ans:} \\

  				\item
  				Sum of all elements in the matrix (use a for/while loop) \\
\begin{verbatim}
matrixsum <- function (mat) {
  i<-1
  sum<-0
  while(i<=dim(mat)[1]*dim(mat)[2])
  {
    sum<-sum+mat[i]
    i<-i+1
  }
  print (paste(sum, "is the sum of elements"))
}

> matrixsum(x)
[1] "45 is the sum of elements"
\end{verbatim}

			\end{enumerate}
		\end{enumerate}
\end{document}		
